\documentclass[12pt]{article}
\setlength{\oddsidemargin}{0in}
\setlength{\evensidemargin}{0in}
\setlength{\textwidth}{6.5in}
\setlength{\parindent}{0in}
\setlength{\parskip}{\baselineskip}

\usepackage{amsmath,amsfonts,amssymb}
\usepackage{geometry}
\geometry{margin=1in}
\usepackage{changepage}
\usepackage{color}
\usepackage{mdframed}
\usepackage{enumitem}
\newmdenv[
  topline=false,
  bottomline=false,
  rightline=false,
  linecolor=blue,
  linewidth=1.5pt,
  skipabove=\topsep,
  skipbelow=\topsep
]{siderules}


\title{Problems Collection}

\begin{document}
UC Berkeley | Fall 2020 \\
\textbf{160 Stellar Physics} Lecture \hfill Problems Archive
\vspace{-0.5cm}

\hrulefill

The following problems are designed to solidify and extend what you have learned in lecture, and can be used as practice questions for exams. Please show \textbf{all} work and justify your responses.

\section*{Problem 1}
At the center of the Milky Way, we infer the existence of a supermassive black hole because of the orbits observed of stars surrounding this central region. Telescopes have found stars orbiting a supposedly empty region of space with a period of 15.2 years and a semi-major axis of 5.5 light days. What is the inferred mass of the black hole based on these measurements?



\section*{Problem 2}
The filaments in a standard 100 W incandescent light bulb are heated to 2800 K. The filament itself is made of tungsten and has a surface area of $2.5 \times 10^{-5}$ m$^2$. What is the luminosity of the light bulb? How much of this energy is in the visible part of the spectrum? Where does the rest of the energy go?


\section*{Problem 3}
What is the wavelength of the photon emitted when an electron transitions from the $n = 3$ to $n = 2$ energy level of a Bohr helium atom? Please retain at least 4 digits of your answer.


\section*{Problem 4}
The Sun's core pressure has a thermal and radiation component. The radiation pressure integral is given by:
\begin{equation*}
P_{rad} = \frac{1}{3} \int_{0}^{\infty} u_{\nu} ~d\nu
\end{equation*}
where for blackbody radiation the intensity of photons $u_{\nu}$ is given as,
\begin{equation*}
u_{\nu} = \frac{8\pi h\mu^3}{c^3}\frac{1}{e^{h\nu /kT}-1}.
\end{equation*}
Show that by working through the pressure integral above, we can obtain the simplified expression 
\begin{equation*}
P_{rad} = \frac{1}{3}aT^4
\end{equation*}
where $a = 4\sigma/c$ with $\sigma$ being the Stefan-Boltzmann constant.


\section*{Problem 5}
Is there helium fusion in the Sun?


\section*{Problem 6}
Solving the system of stellar structure relations and their constitutive relations typically requires numerical approaches. However, one family of solutions to the stellar structure equations can be solved analytically and these are known as polytropes. The equation that describes analytical stellar models is known as the \textit{Lane-Emden equation} and are highly idealized cases. It is written as
\begin{equation}
\frac{1}{\xi^2}\frac{d}{d\xi}\left[\xi^2 \frac{d\Theta}{d\xi}\right] = -\Theta_n^n
\end{equation}
where $n$ is the polytropic index that specifies the dimensionless function $\Theta_n (\xi)$. With the proper boundary conditions, we can find the $n=0$ solution (there are also solutions for $n=1$ and $n=5$) to be
\begin{equation}
\Theta_0(\xi) = 1-\frac{\xi^2}{6}.
\end{equation}
Show the boundary conditions that must be imposed and derive this result.


\section*{Problem 7}
The Sun's photosphere is its ``surface" which has a characteristic temperature of $T = 5777$ K. Here, there are about 500,000 hydrogen atoms for each calcium atom with an electron pressure $P_e = 1.5$ N m$^{-2}$. Compare the strengths of the hydrogen Balmer absorption line and the Ca II K line. Take the hydrogen ionization energy to be $\chi_I = 13.6$ eV, the Ca I ionization energy as $\chi_I = 6.11$ eV, and the degeneracies at a given energy level $n$ for hydrogen atoms to be $g_n = 2n^2$. Assume the first excited state of Ca II is $3.12$ eV above the ground state, the partition functions are $Z_{I} = 1.32$ and $Z_{II} = 2.30$, and the degeneracies are $g_1 = 2$ and $g_2 = 4$.
\begin{enumerate}[label=(\alph*)]
\item Compute the fraction of hydrogen atoms that are ionized (H II).


\item What fraction of the neutral hydrogen atoms have electrons in the first excited state?




\item Using the two preceding results, find the fraction of hydrogen atoms capable of producing Balmer absorption lines.



\item Find the fraction of ionized Ca II ions, the atoms capable of producing the Ca II K line, by repeating the calculation above.


\item Which line is stronger and why?

\end{enumerate}

\end{document}