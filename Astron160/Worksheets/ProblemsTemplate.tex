\documentclass[12pt]{article}
\setlength{\oddsidemargin}{0in}
\setlength{\evensidemargin}{0in}
\setlength{\textwidth}{6.5in}
\setlength{\parindent}{0in}
\setlength{\parskip}{\baselineskip}

\usepackage{amsmath,amsfonts,amssymb}
\usepackage{geometry}
\geometry{margin=1in}
\usepackage{changepage}
\usepackage{color}
\usepackage{mdframed}
\usepackage{enumitem}
\newmdenv[
  topline=false,
  bottomline=false,
  rightline=false,
  linecolor=blue,
  linewidth=1.5pt,
  skipabove=\topsep,
  skipbelow=\topsep
]{siderules}


\title{Problems Collection}

\begin{document}
UC Berkeley | Fall 2020 \\
\textbf{160 Stellar Physics} Lecture \hfill Problems Archive
\vspace{-0.5cm}

\hrulefill

The following problems are designed to solidify and extend what you have learned in lecture, and can be used as practice questions for exams. Please show \textbf{all} work and justify your responses.

\section*{Problem 1}
At the center of the Milky Way, we infer the existence of a supermassive black hole because of the orbits observed of stars surrounding this central region. Telescopes have found stars orbiting a supposedly empty region of space with a period of 15.2 years and a semi-major axis of 5.5 light days. What is the inferred mass of the black hole based on these measurements?

\vspace{0.5cm}
\begin{adjustwidth}{2.5em}{0pt}
\begin{siderules}
 {\color{blue} \textbf{Solution}
 
 Let's define the variables first and let $M_{BH}$ be the mass of the black hole, $M_{\star}$ be the mass of the star, and $a$ as the semi-major axis. Using Kepler's Third Law, we can write the following:
 \begin{equation}
 M_{BH} + M_{\star} = \frac{4\pi^2 a^3}{GP^2}.
 \end{equation}
 If we take $M_{BH} \gg M_{\star}$ and we express $a \ [=]$ AU and $P \ [=]$ years, then we have 5.5 light days = 952.94 AU. We solve directly for the $M_{BH}$ term.
 \begin{equation}
 M_{BH} = \frac{a^3}{P^2} = \frac{(952.92 \ \text{AU})^3}{(15.2 \ \text{years})^2} = 3.74 \times 10^6 \ \text{solar masses}.
 \end{equation}
 This equates to $7.44 \times 10^{36}$ kg.
 }
\end{siderules}
\end{adjustwidth}

\section*{Problem 2}
The filaments in a standard 100 W incandescent light bulb are heated to 2800 K. The filament itself is made of tungsten and has a surface area of $2.5 \times 10^{-5}$ m$^2$. What is the luminosity of the light bulb? How much of this energy is in the visible part of the spectrum? Where does the rest of the energy go?

\setcounter{equation}{0}
\vspace{0.5cm}
\begin{adjustwidth}{2.5em}{0pt}
\begin{siderules}
 {\color{blue} \textbf{Solution}
 
To find the luminosity of the bulb, use the Stefan-Boltzmann equation.
\begin{equation}
L = A\sigma T^4
\end{equation}
Using $A = 2.5 \times 10^{-5}$ m$^2$, $T = 2800$ K, and $\sigma = 5.6704 \times 10^{-8}$ W m$^{-2}$ K$^{-4}$, we obtain $L = 87.1$ W.
To find the amount of energy radiated in the visible range, we integrate $B_{\nu}(T)$ in the visible frequencies, $4.3 \times 10^{14}$ Hz to $7.5 \times 10^{14}$ Hz.
\begin{equation}
L = A \int_{0}^{2\pi} d\phi \int_{0}^{\pi/2} \cos \theta \sin \theta ~d\theta \int_{0}^{\infty} B_{\nu}(T) ~d\nu
\end{equation}
\begin{equation}
B_{\nu}(T) = \frac{2h\nu^3}{c^2\left(e^{\frac{h\nu}{kT}}-1 \right)}
\end{equation}
Substituting this term and simplifying Eq. 2, we obtain the following non-analytical integral:
\begin{equation}
L = A\pi \int_{4.3\times 10^{14}}^{7.5\times 10^{14}} \frac{2h\nu^3}{c^2\left(e^{\frac{h\nu}{kT}}-1 \right)} ~d\nu
\end{equation}
which we can solve numerically using any program. A standard \texttt{SciPy} routine evaluates this as 5.08 W.

This means that out of the 87.1 W radiating from the light bulb, only 5.8\% is in the visible light part of the spectrum. The rest of the light bulb's energy is dissipated in other frequencies of the spectrum, particularly the infrared which we can perceive as heat, because the peak emission occurs in the infrared regime. We can show this by Wien's displacement law.
\begin{equation}
\nu_{max} = (5.879 \times 10^{10} ~\text{Hz/K})(2800 ~\text{K}) = 1.65 \times 10^{14} ~\text{Hz}
\end{equation}
 }
\end{siderules}
\end{adjustwidth}

\section*{Problem 3}
What is the wavelength of the photon emitted when an electron transitions from the $n = 3$ to $n = 2$ energy level of a Bohr helium atom? Please retain at least 4 digits of your answer.

\setcounter{equation}{0}
\vspace{0.5cm}
\begin{adjustwidth}{2.5em}{0pt}
\begin{siderules}
 {\color{blue} \textbf{Solution}
 
Assuming the helium atom is not an ion that is hydrogen-like such as He$^{+}$, then we take the simplistic Bohr model where two electrons in a neutral helium atom are diametrically opposite of each other in a circular orbit. From the electrical and repulsion forces given by Coulomb's law, we have:
\begin{equation}
F_c = \frac{2e^2}{4\pi \epsilon_0 r^2} - \frac{e^2}{4\pi \epsilon_0(2r)^2} = \frac{7e^2}{4\pi \epsilon_0(4r^2)}
\end{equation}
This is equal to the centripetal force given by:
\begin{equation}
F_c = \frac{\mu v^2}{r} = \frac{7e^2}{4\pi \epsilon_0(4r^2)}
\end{equation}
where $\mu$ is the reduced mass. The kinetic energy for each electron immediately follows:
\begin{equation}
K = \frac{1}{2}\mu v^2 = \frac{7e^2}{4\pi \epsilon_0(8r)}
\end{equation}
This means the net kinetic energy in the two-electron system is:
\begin{equation}
K_{net} = \frac{7e^2}{4\pi \epsilon_0(4r)}
\end{equation}
The electrical potential energy is found the same way.
\begin{equation}
U = -\frac{kQq}{r} = \frac{-4e^2}{r(4\pi\epsilon_0)} - \frac{-e^2}{2r(4\pi\epsilon_0)} = -\frac{7}{2}\frac{e^2}{4\pi r\epsilon_0}
\end{equation}
Thus, the total energy is:
\begin{equation}
E_T = U + K = -\frac{7}{2}\frac{e^2}{4\pi r\epsilon_0} + \frac{7e^2}{4\pi \epsilon_0(4r)} = -\frac{7}{4}\frac{e^2}{4\pi r\epsilon_0}
\end{equation}
Using Bohr's quantization of angular momentum:
\begin{equation}
L = \mu vr = n\hbar \Rightarrow v = \frac{n\hbar}{\mu r}
\end{equation}
Solving for kinetic energy from Eq. 3, we have
\begin{equation}
\begin{aligned}
\frac{1}{2}\mu v^2 &= \frac{7}{8} \frac{e^2}{4\pi r \epsilon_0} \\
\frac{1}{2}\mu \left( \frac{n^2\hbar^2}{\mu^2 r^2}\right) &= \frac{7}{8} \frac{e^2}{4\pi r \epsilon_0} \\
\Rightarrow r &= \frac{4n^2\hbar^2(4\pi \epsilon_0)}{7\mu e^2}
\end{aligned}
\end{equation}
Substituting this expression for radius into the expression for total energy, we find that
\begin{equation}
E_T = -\frac{49}{16}\frac{\mu e^4}{n^2\hbar^2(4\pi\epsilon_0)^2}
\end{equation}
When the two electrons make a simultaneous transition, we can find the wavelength of light emitted in the same way as for the hydrogen atom.
\begin{equation}
\begin{aligned}
\frac{hc}{\lambda} &= E_{n_2} - E_{n_1} \\
\frac{hc}{\lambda} &= \frac{49}{16}\frac{\mu e^4}{\hbar^2(4\pi\epsilon_0)^2}\left[ \frac{1}{n_1^2} - \frac{1}{n_2^2} \right] \\
\frac{2\pi \hbar c}{\lambda} &= \frac{49}{16}\frac{\mu e^4}{\hbar^2(4\pi\epsilon_0)^2}\left[ \frac{1}{n_1^2} - \frac{1}{n_2^2} \right] \hfill (\text{note the } \hbar) \\
\frac{1}{\lambda} &= \frac{49}{16}\frac{\mu e^4}{2\pi c \hbar^2(4\pi\epsilon_0)^2}\left[ \frac{1}{n_1^2} - \frac{1}{n_2^2} \right]
\end{aligned}
\end{equation}
With $n_1 = 2$ and $n_2 = 3$, the right side of the above expression yields $6.714 \times 10^7$ m$^{-1}$. Therefore, the transition from $n = 3$ to $n = 2$ of a Bohr helium atom emits a photon of wavelength
\begin{equation}
\lambda = 1.0724 \times 10^{-7} ~\text{m} = 107.24 ~\text{nm}.
\end{equation}
This is in the UV part of the spectrum.
 }
\end{siderules}
\end{adjustwidth}

\section*{Problem 4}
The Sun's core pressure has a thermal and radiation component. The radiation pressure integral is given by:
\begin{equation*}
P_{rad} = \frac{1}{3} \int_{0}^{\infty} u_{\nu} ~d\nu
\end{equation*}
where for blackbody radiation the intensity of photons $u_{\nu}$ is given as,
\begin{equation*}
u_{\nu} = \frac{8\pi h\mu^3}{c^3}\frac{1}{e^{h\nu /kT}-1}.
\end{equation*}
Show that by working through the pressure integral above, we can obtain the simplified expression 
\begin{equation*}
P_{rad} = \frac{1}{3}aT^4
\end{equation*}
where $a = 4\sigma/c$ with $\sigma$ being the Stefan-Boltzmann constant.

\setcounter{equation}{0}
\vspace{0.5cm}
\begin{adjustwidth}{2.5em}{0pt}
\begin{siderules}
 {\color{blue} \textbf{Solution}
 
 
 We first substitute $u_{\nu}$ into $P_{rad}$ and perform a change of variables.
 \begin{equation}
 \begin{aligned}
 P_{rad} &= \frac{1}{3}\int_{0}^{\infty} u_{\nu} ~d\nu = \frac{1}{3}\int_{0}^{\infty} \frac{8\pi h\nu^3}{c^3}\frac{1}{e^{h\nu/kT}-1} ~d\nu \\
 &= \frac{1}{3}\frac{8\pi}{c}\int_{0}^{\infty} \frac{h\nu^3}{c^3}\frac{1}{e^{h\nu/kT}-1} ~d\nu \\
 &= \frac{1}{3}\frac{8\pi}{c}\frac{k^3 T^3}{h^2 c^2} \int_{0}^{\infty} \left(\frac{h\nu}{kT}\right)^3 \frac{1}{e^{h\nu/kT}-1} ~d\nu
 \end{aligned}
 \end{equation}
 Setting $x \equiv \frac{h\nu}{kT}$ and $dx \equiv \frac{h}{kT} d\nu$ for our change of variables gives us
 \begin{equation*}
 \begin{aligned}
 & \frac{1}{3}\frac{8\pi}{c}\frac{k^3 T^3}{h^2 c^2}\frac{kT}{h} \int_{0}^{\infty} \frac{x^3}{e^x - 1} ~dx \\
 = &  \frac{1}{3}\frac{8\pi}{c}\frac{k^3 T^3}{h^2 c^2}\frac{kT}{h} \int_{0}^{\infty} \frac{x^3 e^{-x}}{1-e^{-x}} ~dx .
 \end{aligned}
 \end{equation*}
 We can use the geometric series
 \begin{equation}
 \sum_{n=1}^{\infty}(e^{-x})^n = \frac{e^{-x}}{1-e^{-x}}
 \end{equation}
 to find the analytic solution to the integral via integration by parts. Rewriting our pressure integral, we have
 \begin{equation}
 \frac{1}{3}\frac{8\pi}{c}\frac{k^3 T^3}{h^2 c^2}\frac{kT}{h} \int_{0}^{\infty}  \sum_{n=1}^{\infty} x^3 e^{-nx} ~dx
 \end{equation}
 Setting $U \equiv x^3$, $dv \equiv e^{-nx} dx$, $dU = 3x^2$, and $v = \int e^{-nx} ~dx = -\frac{e^{-nx}}{n}$, we obtain the solution to the integral. Evaluating this gives
 \begin{equation}
 \left. \frac{-x^3 e^{nx}}{n} \right |_0^{\infty} + \int_0^{\infty} \frac{3x^2 e^{-nx}}{n} ~dx = \frac{6}{n^4}.
 \end{equation}
 Placing this result back into the pressure integral gives
 \begin{equation}
 \frac{1}{3}\frac{8\pi}{c}\frac{k^3 T^3}{h^2 c^2}\frac{kT}{h} \sum_{n=1}^{\infty} \frac{6}{n^4}.
 \end{equation}
 We can find the Fourier coefficients and apply Parseval's theorem to evaluate the converging series. We find that $\sum_{n=1}^{\infty} \frac{6}{n^4} = \frac{\pi^4}{15}$ and putting the terms together, we now have
 \begin{equation}
  \frac{1}{3}\frac{8\pi}{c}\frac{k^3 T^3}{h^2 c^2}\frac{kT}{h} \frac{\pi^4}{15} = \frac{4}{3c}\left(\frac{2\pi^5 k^4}{15h^3 c^2}\right) T^4.
 \end{equation}
 It turns out that the Stefan-Boltzmann constant is decomposed as $\sigma = \frac{2\pi^5 k^4}{15h^3 c^2}$, so we obtain the desired simplification:
 \begin{equation}
 \frac{4}{3c}\sigma T^4 = \frac{1}{3}aT^4
 \end{equation}
 where we had defined $a = 4\sigma/c$.
 }
\end{siderules}
\end{adjustwidth}

\section*{Problem 5}
Is there helium fusion in the Sun?

\setcounter{equation}{0}
\vspace{0.5cm}
\begin{adjustwidth}{2.5em}{0pt}
\begin{siderules}
 {\color{blue} \textbf{Solution}
 
 Using the quantum mechanical estimate of the temperature required for reaction to occur:
 \begin{equation}
 T_{quantum} = \frac{Z_1^2 Z_2^2 e^4 \mu_m}{12\pi^2 \epsilon_0^2 h^2 k}
 \end{equation}
 For the collision of two protons, this is approximately $10^7$ K. Helium has two protons and the fusion of two helium nuclei would result in $Z_1 = Z_2 = 2$. The reduced mass for the collision of two protons is $\mu_m = m_p/2$ but with two helium atoms, this becomes $\mu_m = m_p$. From here, the argument is just a matter of scaling.
 \begin{equation}
  T_{quantum} = (2^2 2^2 2)\cdot 10^7 = 3.2\times 10^8 ~\text{K}  \sim 10^8 ~\text{K}
 \end{equation}
 The required temperature for helium fusion to occur would be upwards of $10^8$ K, so we can conclude that there is not helium fusion occurring in the Sun because this would require the temperature of the core to be much hotter than the $10^7$ K that it is at currently.
 }
\end{siderules}
\end{adjustwidth}

\section*{Problem 6}
Solving the system of stellar structure relations and their constitutive relations typically requires numerical approaches. However, one family of solutions to the stellar structure equations can be solved analytically and these are known as polytropes. The equation that describes analytical stellar models is known as the \textit{Lane-Emden equation} and are highly idealized cases. It is written as
\begin{equation}
\frac{1}{\xi^2}\frac{d}{d\xi}\left[\xi^2 \frac{d\Theta}{d\xi}\right] = -\Theta_n^n
\end{equation}
where $n$ is the polytropic index that specifies the dimensionless function $\Theta_n (\xi)$. With the proper boundary conditions, we can find the $n=0$ solution (there are also solutions for $n=1$ and $n=5$) to be
\begin{equation}
\Theta_0(\xi) = 1-\frac{\xi^2}{6}.
\end{equation}
Show the boundary conditions that must be imposed and derive this result.

\setcounter{equation}{0}
\vspace{0.5cm}
\begin{adjustwidth}{2.5em}{0pt}
\begin{siderules}
 {\color{blue} \textbf{Solution}
 
Start with the Lane-Emden equation.
\begin{equation*}
\frac{1}{\xi^2}\frac{d}{d\xi}\left[\xi^2 \frac{d\Theta}{d\xi}\right] = -\Theta_n^n
\end{equation*}
To solve this second-order differential equation, two boundary conditions must be imposed. The central density of the star is defined by
\begin{equation}
\rho(\xi) = 1 \rightarrow \Theta(0) = 1
\end{equation}
This is essentially a normalization of the density scaling function $\Theta$. The second boundary is the central boundary condition, which says that there is no mass inside zero radius. This means that for the density with respect to the radius
\begin{equation}
\frac{d\rho}{dr} \rightarrow 0 ~as ~ r \rightarrow 0
\end{equation}
Since the hydrostatic equilibrium equation $\frac{dP}{dr}$ is proportional to $\frac{d\Theta}{d\xi}$ and depends on density $\rho$ and the equation of state $P_n (r) = K\rho_n^{(n+1)/n}$ for the polytropic model, then the above condition directly leads to
\begin{equation}
\frac{d\Theta}{d\xi} = 0 ~at ~ \xi = 0.
\end{equation}
For the $n=0$ analytic solution to the Lane-Emden equation, we have
\begin{equation}
\frac{1}{\xi^2}\frac{d}{d\xi}\left[\xi^2 \frac{d\Theta}{d\xi}\right] = -1
\end{equation}
Separating the variables and integrating:
\begin{equation}
\begin{aligned}
\int \frac{d}{d\xi} \left[\xi^2 \frac{d\Theta}{d\xi} \right] ~d\xi &= -\int \xi^2 ~d\xi \\
\xi^2 \frac{d\Theta}{d\xi} &= -\frac{\xi^3}{3} + C_1 \\
\frac{d\Theta}{d\xi} &= -\frac{\xi}{3} + \frac{C_1}{\xi^2}
\end{aligned}
\end{equation}
We can now apply the central boundary condition. Since $\frac{d\Theta}{d\xi} = 0$ when $\xi = 0$, we obtain
\begin{equation}
\begin{aligned}
C_1 &= 0 \\
\Rightarrow & \frac{d\Theta}{d\xi} = -\frac{\xi}{3}
\end{aligned}
\end{equation}
Integrating, we find the dimensionless function and a second constant:
\begin{equation}
\Theta(\xi) = -\frac{\xi^2}{6} + C_2
\end{equation}
Applying the boundary condition for density at the center of the star:
\begin{equation*}
\Theta(0) = 0 + C_2 = 1
\end{equation*}
This yields the solution of
\begin{equation}
\Theta(\xi) = -\frac{\xi^2}{6} + 1 = 1 - \frac{\xi^2}{6}
\end{equation}
which is what we expected.
 }
\end{siderules}
\end{adjustwidth}

\section*{Problem 7}
The Sun's photosphere is its ``surface" which has a characteristic temperature of $T = 5777$ K. Here, there are about 500,000 hydrogen atoms for each calcium atom with an electron pressure $P_e = 1.5$ N m$^{-2}$. Compare the strengths of the hydrogen Balmer absorption line and the Ca II K line. Take the hydrogen ionization energy to be $\chi_I = 13.6$ eV, the Ca I ionization energy as $\chi_I = 6.11$ eV, and the degeneracies at a given energy level $n$ for hydrogen atoms to be $g_n = 2n^2$. Assume the first excited state of Ca II is $3.12$ eV above the ground state, the partition functions are $Z_{I} = 1.32$ and $Z_{II} = 2.30$, and the degeneracies are $g_1 = 2$ and $g_2 = 4$.
\begin{enumerate}[label=(\alph*)]
\item Compute the fraction of hydrogen atoms that are ionized (H II).
\setcounter{equation}{0}
\vspace{0.5cm}
\begin{adjustwidth}{2.5em}{0pt}
\begin{siderules}
 {\color{blue} \textbf{Solution}
 
We must first compute the fraction of hydrogen atoms that are ionized by using the Saha equation.
\begin{equation}
\left[\frac{N_{II}}{N_I}\right]_H = \frac{2kTZ_{II}}{P_e Z_I}\left(\frac{2\pi m_e kT}{h^2}\right)^{3/2} e^{-\chi_I/kT}
\end{equation}
These are all constants with the notable exception of the partition functions $Z_I$ and $Z_{II}$. How do we find these? Recall the definition of the partition function $Z$ (this is Eq. 8.7 in Carroll \& Ostlie). Since a hydrogen ion is a proton, it has no degeneracy and thus $Z_{II} = 1$. For $Z_I$, since the energy of the first hydrogen excited state is 10.2 eV above the ground state, it is much larger than $kT$ and thus, the term $e^{-(E_2 - E_1)/kT}$ is very small. Therefore, the partition function can be approximated as $Z_1 \sim g_1$. Since $g_n = 2n^2$, then $g_1 = 2$ and we can take $Z_I \approx 2$.

Once we compute the value of the Saha equation with these constants, we find a fraction of about 1/13,358. This means that there is only one hydrogen ion for every $\sim 13,000$ neutral hydrogen atoms.
 }
\end{siderules}
\end{adjustwidth}

\item What fraction of the neutral hydrogen atoms have electrons in the first excited state?


\vspace{0.5cm}
\begin{adjustwidth}{2.5em}{0pt}
\begin{siderules}
 {\color{blue} \textbf{Solution}
 
For atoms of a given element and ionization state, the ratio of the number of atoms $N_b$ with energy $E_b$ to the number of atoms $N_a$ with energy $E_a$ in different excitation states is found with the Boltzmann equation. The statistical weights of the energy levels are $g_1 = 2$ and $g_2 = 8$ for hydrogen. The difference between $E_2$ and $E_1$ is 10.2 eV.
\begin{equation}
\left[\frac{N_2}{N_1}\right]_H = \frac{g_2}{g_1} e^{-(E_2 - E_1)/kT}
\end{equation}
Evaluating this expression gives about $4.956 \times 10^{-9} \approx 1/201,750,522$.
 }
\end{siderules}
\end{adjustwidth}

\item Using the two preceding results, find the fraction of hydrogen atoms capable of producing Balmer absorption lines.

\vspace{0.5cm}
\begin{adjustwidth}{2.5em}{0pt}
\begin{siderules}
 {\color{blue} \textbf{Solution}
 
The only hydrogen atoms that can produce Balmer absorption lines are those already in the first excited state.
\begin{equation}
\frac{N_2}{N_{total}} = \left(\frac{N_2}{N_1 + N_2}\right)\left(\frac{N_I}{N_{total}}\right) = \left(\frac{N_2/N_1}{1+N_2/N_1}\right)\left(\frac{1}{1+N_{II}/N_I}\right)
\end{equation}
This gives approximately 1/201,765,626, meaning that one of more than 200 million hydrogen atoms is in the first excited state, capable of producing Balmer absorption lines.
 }
\end{siderules}
\end{adjustwidth}

\item Find the fraction of ionized Ca II ions, the atoms capable of producing the Ca II K line, by repeating the calculation above.

\vspace{0.5cm}
\begin{adjustwidth}{2.5em}{0pt}
\begin{siderules}
 {\color{blue} \textbf{Solution}
 
Following the steps above, we first compute the fraction of calcium atoms in the Ca II state. The partition functions are given as $Z_I = 1.32$ and $Z_{II} = 2.30$, with a Ca I ionization energy of $\chi_I = 6.11$ eV. Using the Saha equation:
\begin{equation*}
\left[\frac{N_{II}}{N_I}\right]_{Ca} = \frac{2kTZ_{II}}{P_e Z_I}\left(\frac{2\pi m_e kT}{h^2}\right)^{3/2} e^{-\chi_I/kT} = 905.4
\end{equation*}
meaning that one calcium atom of more than 900 are in the neutral state. The vast majority of atoms are in the Ca II state.

Now we consider the ratio of calcium ions in the excited to the ground state. The energy difference between this first excited state and ground state is 3.12 eV as given, with degeneracies of $g_1 = 2$ and $g_2 = 4$.
\begin{equation*}
\left[\frac{N_2}{N_1}\right]_{Ca II} = \frac{g_2}{g_1} e^{-(E_2 - E_1)/kT} \approx \frac{1}{265}
\end{equation*}
This means that one Ca II ion is in the excited state out of every 265 atoms and are capable of producing the Ca II K line. Now, we can see that almost all of the calcium atoms in the photosphere are in the ground state by computing its fraction:
\begin{equation}
\frac{N_1}{N_{total}} = \left(\frac{1}{1+[N_2/N_1]_{Ca II}}\right)\left(\frac{[N_{II}/N_I]_{Ca II}}{1+[N_{II}/N_I]_{Ca II}}\right) = 0.995
\end{equation}
This indicates that the vast majority of calcium atoms are in the ground state.
 }
\end{siderules}
\end{adjustwidth}

\item Which line is stronger and why?

\vspace{0.5cm}
\begin{adjustwidth}{2.5em}{0pt}
\begin{siderules}
 {\color{blue} \textbf{Solution}
 
There are 500,000 hydrogen atoms for every calcium atom in the solar photosphere. Only a fraction of them ($\sim 1/200,000,000$) are in the first excited state and can produce Balmer lines as shown in (b). What fraction of 500,000 can produce the Balmer line?
\begin{equation}
(500,000)(4.956 \times 10^{-9}) \approx 1/403
\end{equation}
Since we have shown in (c) that nearly all calcium atoms are capable of forming the K line, this means that there are about 400 times \emph{more} Ca II ions with electrons in the ground state to produce the K line than there are in neutral hydrogen atoms with electrons in the first excited state to produce the Balmer line. Consequently, the Ca II K line is much stronger than the Balmer lines in the Sun's spectrum.
 }
\end{siderules}
\end{adjustwidth}
\end{enumerate}

\end{document}